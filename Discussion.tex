\subsection{Threats of Validity}\label{sec:discussion}


The value of this study is subjected to a number of validity threats. We classify the different types of threats of validity as described in~\cite{Wohlin2000}. %\todo{(Wohlin et al. 2000)}.

\noindent {\bf Internal validity}. % threats are concerned with how the treatment influences the outcome. 
There is a threat of internal validity in regards of selecting the right people to interview. To alleviate this issue, we carefully selected different profiles by exploiting also the knowledge and the experience of our contact point within the company. %whenever we suspected the interviewees could not provide relevant for our study information, we asked if they could suggest some other stakeholder. 
Additionally, the interview questions might not have been understood clearly by the interviewees. To mitigate this risk we performed a group interview with stakeholders from the service center, the old system, and the new system. We also hoped that in this way we could encourage these stakeholders to elaborate more on the problem under study. Furthermore, the limited time to carry this study only allowed us to focus mostly on a single interface and limit the number of times we could evaluate the suggested methods. 


%Construct validity threats are concerned with the design of the study.
\noindent {\bf Construct validity}. % threats are concerned with the theory and the results of the study. 
%The number of customer specific use cases were not clear. 
%The stakeholders had rather hard time to remember concrete variability issues in the old system. % and there was only one customer in the new system by the time our study was performed.
%It is therefore not clear whether or not 
There is indeed some uncertainty on whether our suggestion will provide adequate support to manage all the future use cases in the long run. To alleviate this threat, we dealt with identified use cases individually. In this way, we strove to understand their characteristics and based on these to suggest ways to manage them. We document the decision making process and we suggest future work to further improve this study.

\noindent {\bf External validity}. %threats are concerned with the generalizability of the findings beyond the scope of the study. 
This study was placed only on one company, located in Gothenburg, Sweden. The way this company handles variability is different than other companies who follow traditional software product lines. Therefore, the methods might not be easily replicated to other industrial contexts. The company however interoperates with several other companies and we focus on external interoperability. Moreover, our solution is based on best practices available in the Literature. %Additionally, the study mainly focuses only on one interface which has a few known variation points. To alleviate this issue, expanded the scope of the study a little bit to investigate and suggest ways on a higher level to manage variability issues on another interface. 

\noindent {\bf Conclusion validity}. % threats are concerned with the ability to draw the correct conclusions. 
The conclusion whether or not the outcome of this study is a better approach than the current one the company is using has not been proven yet through integration. However, the evaluation is based on feedback we received by various company's stakeholders, as detailed in Section~\ref{sec:results}. 
%The preliminary evaluation involved an engineer of the new system and a person working on the service center. Additionally, during the last presentation, where we presented the refined version of our approach and discussion about supporting the second interface, we involved more stakeholders. These stakeholders were a line manager, two software engineers of the new system, a system architect involved in the development of the new system, a systems expert of the implementation department and a system architect of the core technical development who is involved in both the new and the old system. The feedback was mostly positive. Although we had a bigger list of people to attend our presentation, not all of them could be present at that time due to work obligations. 
Finally, to reduce the risk of objectivity of the feedback received during the last presentation, we asked to state their reasons why they think something is a good or bad idea. Additionally, we encourage stakeholders to communicate with each other in order to hear a different perspective. %We noticed before that sometimes the received feedback varied depending on individuals  perspective. 
We therefore hope the subjectivity of the feedback was reduced by enabling stakeholders to openly state their opinion and hear other people's opinion as well.


